\documentclass[a4paper, 11pt]{article}
%\documentclass[10pt,a4paper,twocolumn]{article}
\usepackage{amssymb,amsmath}
\usepackage{graphicx}
%\usepackage{times}
%\usepackage{float}
\usepackage{algorithmic}

\title{Distributed Peer-to-peer Filesystem}  
\author{Darius Scerbavicius} 
\date{4th April 2013}

\begin{document}

\maketitle

\begin{abstract}
Although the popularity of services similar to DropBox and Google Drive proves that 
users find such cloud storage file hosting services useful,
currently no decentralised and distributed peer-to-peer filesystem exists that could provide
a service of similar quality but without dependence on any single vendor.
In fact, one of the identified problems in peer-to-peer filesystem research is the lack of a system that could be deployed in real world situations. 
This project attempts to implement such a system,
and to provide a comparison with other systems that have similar goals, by using common
filesystem benchmarks. 

An environment for running experiments on such a filesystem has been setup, with preliminary results showing it is suitable for testing such a filesystem.
A prototype of the filesystem is currently being worked on.

\end{abstract}

\section{Introduction}

A number of companies are providing a remote synchronized directory service, with the most famous examples being Dropbox \cite{dropbox} and Google Drive \cite{gdrive}. These services are not deployed in a peer-to-peer manner -- the user's files are stored on the servers of the company which provides the services. This means, that if the company ever goes out of business, decides to stop offering the service, or simply changes the terms of the service to ones that the user disagrees with, the user will no longer be able to use the service. 

During the last 15 years, a lot of promising technology has been developed that can eliminate the need for 3rd party providers while offering the same high quality service. 
Systems based on a Distributed Hash Table (DHT) have allowed lookup services to be implemented in a completely decentralised network, allowing any participating nodes to efficiently retrieve data. Implementations of DHTs are being successfully used in various peer-to-peer Internet services, such as BitTorrent and Coral Content Distribution Network \cite{coral}.

Such technology is just as well applicable to distributed filesystems. Attempts to implement such systems have been made, with the most prominent names being CFS \cite{cfs}, Pastis \cite{pastis}, OceanStore \cite{oceanstore}, Ivy \cite{ivy}, Infinit \cite{towards}. Except for Freenet \cite{freenet}, which has a completely different focus on anonymity, all of these filesystems were designed for academic research, with performance analyses based mostly on simulations, or otherwise controlled unrealistic environments, and so far no implementations are being used outside the research community by the wider public.

In fact one of the main open problems identified by a survey of such file systems, is the lack of performance analyses on real world deployments of P2P file systems \cite{surveyofopenproblems}, as even in the cases where the experiments were executed without a simulator \cite{infinit}, the characteristics of environments used for testing were different from what is expected to happen in the real world. 

The purpose of this MInf dissertation is to build a Distributed Peer-to-peer File System, capable of performing similar function to the proprietary services such as DropBox \cite{dropbox}, but without depending on any specific vendor, so that the contents of the person's private folder are encrypted and then distributed among other users of the system. The implementation would be open source, and make use of efficient, well-tested algorithms, and open source libraries, and would be produced in such a way that users could easily deploy it on their machines. This would address the problem that no distributed P2P filesystems have so far been adopted in real world situations, and therefore have not been exposed to actual real world problems. The availability of such a project would hopefully stimulate further research.

Although the goal of the project is to produce a real world implementation, an environment for testing the prototype must still exist. However, instead of using a network simulator, the goal is to use a light-weight virtualisation solution, LXC Linux containers to run nodes of the system. This will also allow for comparing this filesystem with other peer-to-peer distributed filesystems.

% Approach
% Unaddressed problems
% File persistence (has this been addressed in any way?)
% Security (encrypted file contents)
% Reading files larger than there currently is space available
% index!!!
%

\section{Project Overview}

Currently there is no clear consensus in the literature on what distributed file system components lead to achieving the best possible performance (why?) In fact the most common benchmark used to test such file systems, the Andrew benchmark, has itself received a lot of criticism \cite{kernelb} due to the workload not being realistic, and due to few I/O operations leading to a lack of meaningful results. Nevertheless, it is still considered the standard benchmark for filesystem performance \cite{kernelb}. Designing a superior benchmark is however out of scope of this project, and would in fact prevent any attempt to compare this file system with other file systems.

The following section will present a complete overview of the project, starting from the lowest protocol level and ending with the design of the frontend. 

\subsection{Key-Based Routing}

The lowest level of the distributed filesystem will be built on a key-based routing protocol, Chord. The purpose of this level is, given a key and a message, to route the message to the node with an identifier closest to the key. In our case, the key is the SHA-1 hash of the file path, and the identifier is the SHA-1 hash of the IP address of the node. The distance between the node's identifier and the key of the message is determined by taking the difference between them. %(talk about chord's routing algorithm like in CFS paper)

Chord can be imagined as a network of nodes connected in a ring, where each node is aware of its neighbours: its successor (the next node in clockwise direction) and its predecessor (next node in counter-clockwise direction). Consistent hashing is used to assign keys to nodes, which balances the load on the network. The value of a key-value pair is assigned to the first node that has an identifier equal to the key, or an identifier that follows the key. 

Nodes maintain a routing table that accelerates lookups. In a system of $N$ nodes, a lookup can be resolved using $O(log N)$ messages to other nodes, suggesting high performance, suitable for a peer-to-peer filesystem \cite{chord}.


% what are the main reasons to use chord instead of other KBRs?

%(how does chord scale?)

\subsection{Distributed Hash Table}

On top of this lowest level, sits the DHT (distributed hash table). Like a hash table, a DHT provides a mapping from keys to values. It allows inserting a value with a certain key, and allows the value to be retrieved, by providing the key that was used to insert it. The DHT operates on a network of nodes, and uses the key-based routing protocol to determine the sought node. The canonical DHT for the Chord protocol is the CFS DHash, which the following implementation will be based on. (how is yours different?)

\subsubsection{Operations}
To b
In order to be suitable for usage in a distributed filesystem, The DHT must provide an interface supporting the following operations:

\begin{itemize}
\item insert(key, value)

The DHT will use Chord to determine which node the key value pair should be stored on. To ensure filesystem persistence, replicas of the value will be stored on $k$ number of nodes coming after the successor of the picked node. It should be noted that the number $k$ has not yet been determined, and a number of experiments will be executed to find a suitable value.
%TODO: describe experiments

\item retrieve(key)

Again, the DHT will use Chord to determine the node carrying the value given by the key. As $k$ number of nodes following that node store a replica of the requested value, the latency to each of them will be measured, and the fastest node will be picked for downloading.

\item delete(key)

Deletion will be implemented simply by deleting the file path from the user's file record (which will be discussed further). To an extent, this also guards against malicious injections of large amounts of data \cite{cfs}, as such data will be gradually deleted.
\end{itemize}
%TODO: address caching

The filesystem may be deployed on storage drives of varying size, including ones that provide less space than some of the files stored on the world-wide filesystem. 

%TODO: describe how CFS does this

\subsubsection{Storage}

\subsection{Versioning}
Due to the way replication works, not all copies of the file currently in the network are going to be up-to-date. To ensure that the user always receives the most recent copy, when performing a lookup, a weighted voting scheme is going to be introduced. The replicas will be assigned a version number, and each replica of a file will be assigned a certain number of votes. Whenever the file is accessed, or written to, a certain number of $r$ votes to read a file, and a certain number of $w$ votes to write a file will be collected, such that $r + w$ is more than the total number of votes assigned to file. This guarantees that any lookup will always retrieve the most recent version of the file \cite{versioning}.

\subsection{Security}

The prototype implementation will not have a sophisticated access control system. A simple mechanism based on a hybrid cryptosystem approach will be provided instead.

% why not do sth complicated like Infinit, need a good argument

Each user of the filesystem will have to generate an AES key for encrypting own data, as well as an RSA key pair for authentication.  
Every user will have a copy of the public keys of all users whose files are stored on this specific user's computer.

The asymmetric key encryption mechanism will ensure that integrity is maintained. If a node received a request to update a file currently stored on the node's machine, the node would use the public key associated with that file to construct a challenge, and send it to the requesting node, awaiting a response.  
If the requesting node provided the correct response, the update would be executed. Same applies to download requests.

The symmetric key mechanism will ensure that the user of the current node cannot read the contents of files that belong to other users of the system, as the current user will only posses own AES key.

This hybrid approach using both symmetric and asymmetric algorithms is necessary to ensure good performance, as asymmetric key algorithms are much more computationally expensive than symmetric key algorithms.


\section{Frontend}

\subsection{Index}
The list of files stored on the filesystem by the user, will be governed by an index. As in CFS, this index will be stored under the user's RSA public key in the distributed hash table. The current plan is to maintain the index as a plain text file, although this is obviously inadequate for filesystems containing large numbers of files. A more sophisticated approach, involving a database, will be presented later. 

Whenever a user creates a new file, a record of the file will be added to the index, and likewise the file's record will be deleted when the user deletes that particular file.

To connect to the system, a user must first obtain an IP address of a node already in the system.

\subsection{FUSE}
The front-end of the filesystem will be based on FUSE, which allows implementing a filesystem as a user space program \cite{fuse}. This would bypass the operating system kernel, leading to greater portability, as FUSE is available for Linux, FreeBSD, NetBSD, OpenSolaris and Mac OS X. FUSE is suitable for long-term projects, as it is well-maintained, and is an official component of the Linux kernel.


\section{Evaluation}

\subsection{Environment}

The evaluation testbed and methods were chosen after a careful assessment of a number of different procedures, under the criteria that:
\begin{enumerate}
\item no significant differences would be required between the implementation used in the simulation, and the implementation used in the real world;
\item the simulation would be capable of running a large amount of nodes ($>$ 2000);
\item there would be a good basis to believe that the obtained results are applicable to the real world.
\end{enumerate}

\subsubsection{ns-3}
The first method considered, was to target ns-3 for the test bed environment, as it has previously been used for this purpose when testing the Infinit filesystem. This method was attempted at first, but it soon became evident that it has a drawback: the filesystem implementation used for simulation would differ too much from the implementation targeting the real world. %This was the case for the Infinit project, as the implementation used in the simulated environment produced  

% TODO: explain how ns3 could be used to do this

\subsubsection{LXC Containers}

To ensure that all criteria are met, it was decided that the filesystem will be tested using a local emulation testbed. The testbed uses Linux Containers (LXC), a high-performance operating system level virtualization solution, that allows sharing the resources of the host machine among many isolated instances. LXC containers use the Linux kernel cgroups feature to segregate processes, isolating them from other processes, and providing them with their own network space. FUSE-based filesystems can be mounted on LXC containers, but the containers can also share certain directories with the host machine.

Such a testbed has previously been used for analyzing the performance of a P2P swarm \cite{p2plxc} using the hrktorrent BitTorrent client, and bttrack BitTorrent tracker, with the conclusion that LXC is suitable for testing P2P network applications in real world scenarios. Each peer was running in its own container, on top of the hardware node, sharing parts of the system with the host, and very little decrease in download speed was noted when the number of peers was gradually increased from 20 to 100. This is surprising, as a very low-end machine was used for executing the tests (Intel(R) Core(TM)2 Duo CPU T7500 @ 2.20GHz, 1.5 GB of RAM), indicating that the LXC containers have a very low overhead, and running a large amount of nodes is possible.

Nevertheless, some of the benchmarks will require up to 10000 nodes at a time. Even a powerful machine may not be able to handle such a large number of nodes running the P2P filesystem. As every LXC container has its own network device configuration, an external IP address can be assigned to each of them, thus exposing the containers to a wider network. The plan is then to have a number of machines running LXC containers, each contributing a different amount of peers to the filesystem network, depending on the machine's resources.

\subsection{Benchmarks}

\subsubsection{Andrew benchmark}

The standard method to evaluate performance in the world of distributed filesystems is the Andrew benchmark \cite{andrewscale}, consisting of:
\begin{enumerate}
\item creating a directory hierarchy;
\item copying files into the created directories;
\item walking the whole directory hierarchy and reading the attributes of files;
\item reading the contents of files;
\item compiling the files into a program.
\end{enumerate}
% topologies?
The filesystem will be accessible using FUSE, so to execute the benchmark, the OpenSSL-1.0.0 library source code will be placed on the filesystem. This is a large-scale open source project, consisting of approximately 1100 source files, and around 700 object files are generated during compilation. Using this OpenSSL library will allow for direct comparison with Infinit \cite{towards}.

In order to produce realistic results that could be compared to other filesystems, a network composed of 16 nodes (the standard amount in distributed filesystem evaluations) located in the same LAN will be setup, and the Andrew benchmark will then be executed on the filesystem. The time required to execute each of the 5 steps of the benchmark will be noted, and compared against Infinit, Pastis, Ivy and others.

The benchmark will then be executed with a gradually increasing amount of nodes: 50, 100, 500, 1000, 5000, 10000. The times obtained will again be compared against other filesystems.

\section{Current Progress}

Most work done so far has been on preparing a suitable testing environment for testing the filesystem.

The first attempt consisted of implementing a subset of Chord (without support for the routing tables), and a simplistic DHT layer on top of it, using ns-3. As ns-3 has support for Python bindings, it seemed as the language of choice for its ease of use compared to C++, but after taking further steps, the Python binding documentation turned out to be very incomplete, and some important parts of the API, such as callbacks, turned out not to be properly supported in Python.  

This instigated a switch to using C++ for development, using the proper C++ API of ns-3. However, as the development of the Chord protocol progressed, it became clear that, although the parts of the ns-3 API relevant to this project are similar to the Unix sockets API (which is in fact one of the main selling points of ns-3), they are not entirely alike, leading to fairly large differences between the prototype used for simulation, and the prototype that would be used in realistic environments.

This led to a search for a different testing environment, concluding with the choice of LXC \cite{p2plxc}. As no user-friendly tool exists to operate LXC containers, a number of shell and Python scripts were written for starting and configuring a large amount of nodes. 

To test the emulation testbed and determine how many nodes a VirtualBox virtual machine running Ubuntu Linux on an Intel(R) Core(TM) i3 CPU M370 @ 2.40GHz, 2 GB of RAM computer could manage at once, an experiment was setup. 
A Python application capable of sending and receiving files was implemented using the Twisted networking library.
Using this application, each pair of the nodes exchanged 30 MB files over the network, and CPU usage was monitored during the exchange, while the number of nodes was gradually increased. The results are visible in figure~\ref{cpu_usage}. Each node was running an instance of a Python process, requiring on average 0.3-0.7\% CPU. As the number of nodes was increased, there was a fairly linear increase in CPU usage, with 2 nodes needing 1.2\% CPU, 10 nodes requiring 6\% CPU, 50 nodes requiring 32\% CPU, 100 nodes taking up $46\%$ CPU and 180 nodes taking up 87\% CPU. Thus the machine is able to support up to 180 LXC containers running at once, as increasing the number of working nodes at this point could start to interfere with other processes running on the system (such as the desktop environment).

Considering the experiment was executed on a virtual machine, this is a fairly large number of nodes, and a non-virtualized instance of Linux is expected to be able to run an even larger amount of nodes in parallel.

\begin{figure}
\begin{center}
  \includegraphics[width=11cm]{cpu_nodes.png}
\end{center}
\caption{CPU usage with increasing number of nodes}
\label{cpu_usage}
\end{figure}  


\section{Future Plans}

The current goal of the project is to get the filesystem to a state where the ANDREW benchmark can be executed on it. At that point it will be possible to compare the speed of the filesystem to other filesystems, and this will make it clearer as to how well it scales when nodes are added/removed, and what its most significant short-comings are, in regards to other similar filesystems.

\subsection{Work Plan}

See figure~\ref{milestones}.
\begin{figure}
\begin{tabular}{ | p{0.65\textwidth} | p{0.35\textwidth} | }
\hline
\textbf{Task} & \textbf{Dates} \\
\hline
Exam Period & April 8th 2013 - May 30th 2013 \\
\hline
Finish implementing Chord and DHash for the Twisted Python framework & June 1st 2013 - July 30th 2013 \\ 
\hline
\hline
{\bf Milestone: Chord and DHash layers implemented} & August 1st 2013 \\
\hline
\hline
Implement the simple security scheme on top of the DHash layer & August 1st 2013 - August 15th 2013 \\
\hline
Implement the FUSE interface to the filesystem. & August 15th 2013 - September 15th 2013 \\
\hline
Determine if there are any missing parts in the filesystem that prevent the Andrew benchmark from running, and implement them & September 15th 2013 - September 30th 2013 \\
\hline
\hline
{\bf Milestone: Andrew benchmark running} & 1st October 2013 \\
\hline
\hline
Compare Andrew benchmark results to other similar filesystems, determine if there are any shortcomings and devise a plan on how to overcome them & October 1st 2013 - October 15th 2013 \\
\hline
? & 15th October 2013 - 30th December 2013 \\
\hline
? & 1st January 2014 -  30th January 2014 \\
\hline
? & 1st February 2014 - 28th February 2014 \\
\hline
\hline
{\bf Milestone: Project finished} & 1st February 2014 \\
\hline
\hline
Write dissertation & 1st February, 2014 - 3rd April 2014 \\
\hline
Create presentation & 4th April 2014 - 30th April 2014 \\
\hline
\end{tabular}
\caption{Project timetable}
\label{milestones}
\end{figure}



%SHIT LEFT TO COVER:
%overwriting files?
%
%introduction
%workplan
%
%main problem with infinit: evaluation does not deal with churn

% As such, each user maintains a file list, of all files in
%the system that belong to the user.
%The current approach is to disregard all access control, and solve the issue later.
%
%
%% could copying be implemented in a clever way? how do other filesystems do it?
%\section{Purpose}
%
%The project will strive for:
%\begin{enumerate}
%\item simplicity, so that potential open source contributors find the project easy to grasp, thus enabling them to submit contributions;
%\item efficiency, since users will only consider the filesystem useful if it offers good performance.
%\end{enumerate}
%
%\section{Background}
%
%
%A list of decentralised peer-to-peer filesystems and their drawbacks follows:
%
%\begin{description}

%\section{Methods}
%
%And essentially no fully decentralised filesystem has been adopted by the wider public. 
%To build such a filesystem, a back-end and a front-end will be designed.
%
%The base of the back-end of such a system is a DHT (distributed hash table), capable of executing three commands among all nodes:
%The novelty of the system lies in being able to keep more files in the cloud, than there currently is space available on your physical hard drive. 
%
%This will be done by maintaining a list of the most frequently or recently used files that will be cached in the system, while leaving the rest of the space (that was allocated to the filesystem by the user) for keeping reconds of other people's files that they stored in the cloud.
%
%This should be possible, as the encouraged behaviour for a user of such a filesystem, is to allocate a large amount of hard drive space to it, but it is likely that the user will only use a small amount of that space for his or her own files (which are then going to be uploaded to this 'cloud' anyway).
%
%Describe how quorums are stored in the system,
%
%How we go from FUSE to executing all of the above (check FUSE API)
%
%Protocol needs to be scalable (as there might be a lot of nodes connected, how to address that?), also needs to support churn, as network nodes will move in and out of the network.
%
%Steps to editing a file:
%\begin{enumerate}
%\item user access a file on his/her filesystem
%\item retrieve(key) is called
%\item user edit and saves the file
%\item insert(key, value)
%\item update is propagated according to quorum rules
%\end{enumerate}
%
%Also need to address: security, persistence (how exactly am I adding sth new to the area)
%\begin{itemize}
%\item The back-end will be based on a DHT (Distributed Hash Table). 
%\item The front-end of the filesystem will be based on FUSE (Filesystem in Userspace) \cite{fuse}.
%\end{itemize}
%The implementation will target a real world workload. The ns-3 \cite{ns3} simulator will be used to test the implementation.
%
%\section{Current Progress}
%
%Work done so far consisted of a prototype implementation of the Chord protocol on top of the Python Twisted networking library, and a setting up of a testing infrastructure, capable of determining whether 
%\subsection{Network}
%\subsection{Replication}
%To enable reliable file storage, the peer-to-peer network must have a system of file replication. The most basic implementation is to store the exact same replica of the file at a fixed number of successor nodes. This is in fact the suggested method for higher level applications by Chord authors \cite{chord}.
%
%In the case that this is deemed inadequate during the development process, a more complicated scheme, such as Dynamic Replication could be implemented instead \cite{dhash}.
%

%\subsection{Cryptography}
%
%Unlike other Peer-to-peer systems such as Freenet, the emphasis of this project is not anonymity, but instead simplicity and efficiency. Only a minimal security mechanism will be provided, where the owner's files will be encrypted using a symmetric-key algorithm, with a possible candidate being Twofish (which is being widely used in a lot of products \cite{twofishprod}).
%This will not provide complete anonymity but will prevent unauthorised access to the user's files, as long as the key is kept safe.
%
%\subsection{Access Control}
%
%Providing an access control system is not in the scope of this project, and the task of implementing such a system is left to a higher-level application. However, an API will be provided in order to make this possible. 
%
%Since the contents of the files stored on the filesystem will be encrypted with the owner's key, this will to an extent prevent unauthorised access to users' resources.
%

%\section{Evaluation}
%
%As was mentioned, the implementation targets a real-world setup. However, due to limited resources, the simulator ns-3 \cite{ns3} will be used for specific test cases. This will allow for performing simulated experiments on systems involving a large number of nodes in order to determine how well the system scales, as well as performing real-world experiments with the intent of comparing performance to other well-established filesystems.
%
%\begin{itemize}
%\item Well-established and commonly used benchmarks in filesystem evaluation are the Andrew benchmark \cite{andrew}, and the Linux kernel build \cite{kernelb}. The results obtained from running these benchmarks would be compared to results presented in other filesystem papers.
%
%\item As it is common to compare the performance of peer-to-peer filesystems to the performance of NFS \cite{oceanstore} \cite{ivy} \cite{pastis}, a number of real world experiments would be structured upon simple operations executed on NFS, and a comparison would be produced.
%\end{itemize}
%
%\section{Outputs}
%
%A prototype of the system described will be made available as an open source project under the GPL \cite{gpl} license. The implementation will consist of a front-end and a back-end, providing full usage capabilities on any Linux-based operating system, as well as Mac OS X. Windows will not be supported, due to the lack of a mature port of FUSE. 
%
%The project will be published on a GitHub \cite{github} repository, and open to contributions from the open source community.
%
%\section{Workplan}
%
%The workplan timetable is given below (removed).
%
%As the project progresses, it is likely that changes will be made to the timetable.

\bibliographystyle{IEEEtran}
\begin{thebibliography}{10}
\bibitem{dropbox}
Dropbox. http://www.dropbox.com/
\bibitem{gdrive}
Google Drive. http://drive.google.com/start/
\bibitem{coral}
M. J. Freedman, E. Freudenthal, D. Mazieres.
Democratizing content publication with Coral.
In NSDI, Mar. 2004.
\bibitem{ns3}
ns-3. Discrete-event network simulator. http://www.nsnam.org/
\bibitem{chord} 
I. Stoica, R. Morris, D. Karger, M. F. Kaashoek, and
H. Balakrishnan. Chord: A scalable peer-to-peer lookup
service for internet applications. In Proceedings of the 2001
Conference on Applications, Technologies, Architectures,
and Protocols for Computer Communications, pages
149–160. ACM Press, 2001.
\bibitem{fuse}
FUSE: Filesystem in Userspace. http://fuse.sourceforge.net/ 
\bibitem{cfs}
F. Dabek, M. F. Kaashoek, D. Karger, R. Morris, and I. Stoica. Wide-area
cooperative storage with CFS. In SOSP, Oct. 2001.
\bibitem{oceanstore}
J. Kubiatowicz, D. Bindel, Y. Chen, S. Czerwinski, P. Eaton, D. Geels, R. Gummadi, S. Rhea,
H. Weatherspoon, W. Weimer, C. Wells, and B. Zhao. Oceanstore: An architecture for globalscale persistent store. In Proc. ASPLOS’2000, Cambridge, MA, November 2000.
\bibitem{ivy}
A. Muthitacharoen, R. Morris, T. Gil, and B. Chen. 
Ivy: A read/write peer-to-peer filesystem. 
In Proc. of OSDI, 2002.
\bibitem{towards}
J. Quintard. Towards a worldwide storage infrastructure.
PhD thesis, University of Cambridge, September 2010.
\bibitem{freenet}
The Freenet Project. https://freenetproject.org/
\bibitem{surveyofopenproblems}
R. Hasan, Z. Anwar, W. Yurcik, L. Brumbaugh, and
R. Campbell. A survey of peer-to-peer storage techniques for
distributed file systems. In ITCC �05: Proceedings of the
International Conference on Information Technology: Coding
and Computing (ITCC�05) - Volume II, pages 205�213,
Washington, DC, USA, 2005. IEEE Computer Society.
\bibitem{dhash}
M. Leslie.
Reliable Data Storage in Distributed Hash Tables. Oxford University. 2005.
\bibitem{versioning}
D. K. Gifford. Weighted Voting for Replicated 
Data. In Proceedings of the Seventh ACM 
Symposium on Operating Systems Principles, 
pages 159-159, December 1979
\bibitem{twofishprod}
Bruce Schneier. Products that use Twofish. http://www.schneier.com/twofish-products.html
\bibitem{andrew}
J. H. Howard. An Overview of the Andrew File System.
In Proceedings of the Winter USENIX Technical Confer-
ence, February 1988.
\bibitem{kernelb}
A. Traeger, N. Joukov, C. P. Wright, and E. Zadok. A
Nine Year Study of File System and Storage Benchmarking. ACM Transactions on Storage (TOS), 4(2):25–80,
May 2008.
\bibitem{pastis}
Busca, J. M., Picconi, F., \& Sens, P. (2005). Pastis: A highly-scalable multi-user peer-to-peer filesystem. Euro-Par 2005 Parallel Processing, 644-644.
\bibitem{gpl}
GNU General Public License. http://www.gnu.org/licenses/gpl.html
\bibitem{github}
GitHub. https://github.com/ 
\bibitem{chordalt}
Alternatives to the Chord Protocol. Boston University. http://nislab.bu.edu/sc546/sc441Spring2003/CallaMiraniCHORD/alternatives.html
\bibitem{andrewscale}
J. Howard, M. Kazar, S. Menees, D. Nichols, M. Satyanarayanan, R. Sidebotham, and M. West. Scale and performance in a distributed filesystem. ACM Transactions
on Computer Systems, 6(1), February 1988.
\bibitem{p2plxc}
M. Bardac, R. Deaconescu, and A. M. Florea, ``Scaling Peer-to-Peer
Testing using Linux Containers'', in Proceedings of the 9th RoEduNet
IEEE International Conference, 2010, pp. 287-292.

\end{thebibliography}
\end{document}
