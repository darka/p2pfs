\documentclass[8pt,a4paper]{article}
%\documentclass[10pt,a4paper,twocolumn]{article}
\usepackage{amssymb,amsmath}
\usepackage{graphicx}
%\usepackage{times}
\usepackage{float}
\usepackage{algorithmic}

\title{Distributed Peer-to-peer File System}  
\author{Darius Scerbavicius} 
\date{25th January 2013}

\begin{document}

\maketitle

\begin{abstract}
This is a proposal.
\end{abstract}

\section{Purpose}
The purpose of this MInf dissertation is to build a Distributed Peer-to-peer File System, capable of performing similar function to the propriertary services such as DropBox \cite{dropbox} and Google Drive \cite{gdrive}, but without depending on any specific vendor, with the contents of the person's shared folder being distributed among other users of the system. 

The project consists of developing a backend based on a peer-to-peer distributed hash table algorithm, and a frontend, which would allow users to place and retrieve files from the file system.

\section{Background}

Currently a number of companies are providing a remote synchronized directory service, with the most famous examples being Dropbox and Google Drive. These services are not distributed -- the user's files are stored on the servers of the company which provides the service. This means, that if the company ever goes out of business, decides to stop offering the service, or simply changes the terms of the service to ones that the user disagrees with, the user will no longer be able to use the service. 
Furthermore, the proposed project has the potential to offer much more free space to its users than any of the currently available centralised services.

During the last 15 years, a lot of promising technology has been developed that can eliminate the providing vendors while offering the same high quality service. 
Systems based on a Distributed Hash Table (DHT) have allowed lookup services to be implemented in a completely decentralised network, allowing any participating nodes to efficiently retrieve data. Implementations of DHT's are being successfully used in various peer-to-peer Internet services, such as BitTorrent and Coral Content Distribution Network.

Such technology is just as well applicable to distributed file systems. Attempts to implement such systems have been made:
\begin{itemize}
  \item CFS -- decentralised peer-to-peer storage system, robust and efficient but read-only
  \item Pastis -- distributed file system based on the Pastry lookup protocol, which is substantially more complicated than Chord \cite{chord}.
  \item OceanStore -- not completely decentralised
  \item Wand
  \item Ivy
\end{itemize}
%Chord
%Kademlia
%CFS
% Problems: read-only
%Ivy
%OceanStore
%WAND File System
%BitTorrent
%DropBox
%Google Drive

So far no implementations are being used outside the research community by the wider public, and no large-scale high quality open source projects capable of offering such a service have been established.


\section{Methods}

To build such a file system, a backend and a frontend will be designed.

The backend will be based on a DHT (Distributed Hash Table). 
% Frontend based on FUSE, 
The frontend of the file system will be based on FUSE (Filesystem in Userspace) \cite{fuse}.

\subsection{Network}
The lookup service for the DHT will be provided by the Chord peer-to-peer overlay network \cite{chord}. Main reasons to choose Chord over other P2P services are its simplicity, the fact that it is well-studied already, and its provable correctness and performance will allow for producing a more formally grounded result. %TODO: cite something (those slides?)

Chord can be imagined as a network of nodes connected in a circle, where each node knows its successor (the next node in clockwise direction) and its predecessor (next node in counter-clockwise direction). Consistent hashing is used to assign keys to nodes, which balances the load on the network. A key-value pair is assigned to the first node that has an identifier equal to the key, or an identifier that follows the key. 

Nodes maintain a routing table that accelerates lookups. In a system of $N$ nodes, a lookup can be resolved using $O(log N)$ messages to other nodes, suggesting high performance, suitable for a peer-to-peer file system.

\subsection{Replication}
To enable reliable file storage, the peer-to-peer network must have a system of file replication. The most basic implementation is to store the exact same replica of the file at a fixed number of successor nodes. This has proven 99\% efficacy (FIND A CITATION).
In the case that this is deemed inadequate, during the development process, a more complicated scheme, such as Dynamic Replication could be implemented instead.
%TODO: cite Dynamic Replication paper

\subsection{Versioning}
Due to the way replication works, not all copies of the file currently in the network are going to be up-to-date. To ensure that the user always received the most recent copy, when performing a lookup, a weighted voting scheme is going to be introduced. The replicas will be assigned a version number, and each replica of a file will be assigned a certain number of votes. Whenever the file is accessed, or written to, a certain number of $r$ votes to read a file, and a certain number of $w$ votes to write a file will be collected, such that $r + w$ is more than the total number of votes assigned to file. This guarantees that any lookup will always retrieve the most recent version of the file.


\subsection{Security}

Unlike other Peer-to-peer systems such as Freenode, the emphasis of this project is not security, but instead simplicity and efficiency. Only a minimal security mechanism will be provided, where the owner's files will be encrypted using a symmetric-key algorithm, with current algorithm candidates being AES (Advanced Encryption Standard, established by NIST) and Twofish (which is being widely used in a lot of products).
This will not provide complete anonymity but will prevent unauthorised access to the user's files, as long as the key is kept safe.


\section{Evaluation}

The project will be evaluated using the ns-3 simulator.

\section{Outputs}

A prototype of the system described will be made available as an open source project under the GPL license. The implementation will consist of a frontend and a backend, providing full usage capabilities on any Linux-based operating system, as well as Mac OS X. Windows will not be supported, due to the lack of a mature port of FUSE. 

The project will be published on a GitHub repository, and open to contributions from the open source community.

\section{Workplan}

% TODO: attach image from presentation (in Google Drive MInf directory)

\bibliographystyle{IEEEtran}
\begin{thebibliography}{10}
\bibitem{chord} 
I. Stoica, R. Morris, D. Karger, M. F. Kaashoek, and
H. Balakrishnan. Chord: A scalable peer-to-peer lookup
service for internet applications. In Proceedings of the 2001
Conference on Applications, Technologies, Architectures,
and Protocols for Computer Communications, pages
149–160. ACM Press, 2001.
\bibitem{dropbox}
Dropbox. http://www.dropbox.com/
\bibitem{fuse}
FUSE: Filesystem in Userspace. http://fuse.sourceforge.net/ 
\bibitem{gdrive}
Google Drive. http://drive.google.com/start/
\end{thebibliography}
\end{document}
